\begin{table}[h]
\centering
\caption{Maximum Rewards, Variation, and Best Noise Level for Each Algorithm and Environment}
\label{tab:numerical_results}
\begin{tabular}{|l|c|c|c|c|c|c|c|}
\hline
Algorithm & HalfCheetah & Hopper & Humanoid & HumanoidStandup & Pusher & Reacher & Swimmer \\ \hline
PPO & - & - & - & - & - & - & - \\ \hline
GenPPO & - & - & - & - & - & - & - \\ \hline
TRPO & - & - & - & - & - & - & - \\ \hline
GenTRPO & - & - & - & - & - & - & - \\ \hline
TRPOR & - & - & - & - & - & - & - \\ \hline
TRPOER & - & - & - & - & - & - & - \\ \hline
\end{tabular}
\end{table}
\section{Statistical Significance Against Baseline}
This section evaluates the statistical significance of each algorithm’s performance compared to its baseline (PPO for GenPPO, TRPO for GenTRPO, TRPOR, TRPOER) using a two-sample t-test. The p-value indicates the likelihood that a noise-trained run outperforms the baseline.
\subsection{GenPPO vs. PPO}
\subsection{GenTRPO vs. TRPO}
\subsection{TRPOR vs. TRPO}
\subsection{TRPOER vs. TRPO}
\section{Comparison Statistics}
This section analyzes the probability that GenPPO outperforms PPO and GenTRPO outperforms TRPO based on maximum rewards across all runs, using a binomial test. It also identifies the model and noise configuration most likely to achieve the highest reward per environment.